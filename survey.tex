\documentclass[journal]{IEEEtran}

% TODO: Remove after adding lorem ipsums
\usepackage[english]{babel}
\usepackage{blindtext}
\usepackage{tabu}

% correct bad hyphenation here
\hyphenation{op-tical net-works semi-conduc-tor}

\begin{document}

\title{Analysis of the Security Requirements Related to AAA in the Internet of Things}

\author{Omar~Erminy,~\IEEEmembership{~Technische Universit\"at Darmstadt}
  \thanks{O. Erminy is currently part of the Distributed Software System master program
  of the Technische Universit\"at Darmstadt, Germany.}
}

\markboth{Seminar Telekooperation, Technische Universit\"at Darmstadt, Summer Semester 2015}
{Shell \MakeLowercase{\textit}: Analysis of the Security Requirements Related to AAA in the Internet of Things}

\maketitle

\begin{abstract}
  The \emph{Internet of Things} has become one of the favorite buzzwords in the technology field during the last couple of years. The hype caused by the almost uncountable tentative applications for this technology and the enormous business opportunity they represent has outshone the underlying aspects of security that are needed to support it. The pervasive characteristic of the \emph{IoT}, its strong autonomous behavior and the possibility of handling sensible data and operations makes it require a strong security support to operate. Authentication, Authorization and Accounting (\emph{AAA}) support considers some of the security concerns that are needed in order to provide a good quality of service to the subjects and processes that will make use of \emph{IoT} platforms. This work evaluates \emph{} several concrete and available \emph{IoT} platforms regarding \emph{AAA} to provide a understanding on the security aspects of nowadays \emph{IoT}'s implementations.
\end{abstract}

\begin{IEEEkeywords}
AAA, IoT, Internet of Things, Authentication, Authorization, Accounting, Security.
\end{IEEEkeywords}

\section{Introduction}
  % Indicates the large scale and the upcoming importance of IoT
  \IEEEPARstart{C}{urrent} technology trends in the the fields of information technology and communication networks have been focusing their attention on the so-called \emph{Internet of Things} ---commonly referred to as \emph{IoT}. The term was first used in 1999 to describe the futuristic idea of autonomous computers that were able to gather information from the real world and take particular actions based on it, without any intervention of a human being \cite{Ashton2009}. Nowadays, the further development of Wireless Sensors and Actuators Networks (WS\&AN), as well as the improvement of mobile computing, enables the necessary infrastructure for sustaining a state where everyday objects would interact between each other in order to take smart decisions without the need of the human physical factor.

  Several technology companies and institutes with renowned international reputation, have stated that the \emph{IoT} represents an important business opportunity. Gartner includes it as one of the top ten technology trends of 2013 \cite{Gartner2012}, while Cisco predicts that the Value at Stake of the \emph{IoT} will be \$14,4 for companies and industries in the next ten years \cite{JosephBradley2013}. The public sector will also be affected; traffic and pollution control, among others, will be enhanced under the concept of smart cities. Moreover, the impact of this technology in our daily lives could represent a major milestone.

  % Introduces the term security and explain briefly AAA. Information security has become one of the new trends
  The inherent pervasive characteristic of the \emph{IoT}, as well as its lack of user's consent, raises an alarm regarding the security aspects behind its realization. These networks of thousands or even millions of interconnected autonomous devices will handle a huge amount of sensible data and it is essential that their implementations fulfill certain security requirements in order to avoid unintentional consequences. However, given the heterogeneous nature of the different elements that can conform such networks, interoperability becomes a challenge, hindering the privacy and the security of the infrastructure.

  Authentication, authorization and accounting ---also known as \emph{AAA}--- represent three relevant security concerns for access control to resources belonging to a network. Efforts on creating a standardized architecture for \emph{AAA} have been carried out by the \emph{Internet Engineering Task Force} and have been enforced \cite{RFC2903}, \cite{RFC2904} although implementations ---specially in the \emph{IoT} context--- are still far from reaching homogeneity.
   
  % Explains the motivation of this survey and its approach  
  In this survey I will carry out an evaluation of three available \emph{IoT} platforms, under different security aspects focusing on \emph{AAA}. The chosen platforms were developed as part of the \emph{Seventh Framework Programme for Research and Technological Development} of the \emph{European Commission} \cite{FP7}. The goal is to indicate if the security concerns were met and determine up until which extent are they being enforced. Additionally, a comparison between the findings is provided. The claims included in this document are based on a theoretical approach, given that not all of the selected case studies offered open access to their source codes nor their usage.

  % Outlines the structure of the paper
  This document is structured as follows: Section \uppercase\expandafter{\romannumeral 2} defines the selected security criteria used to evaluate the chosen platforms. Section \uppercase\expandafter{\romannumeral 3} contains the case studies. It is divided into several subsections, each of which explains the goal of a platform and the way the security concerns are dealt within. Section \uppercase\expandafter{\romannumeral 4} summarizes the evaluation of the three subjects. Finally, section \uppercase\expandafter{\romannumeral 5} offers a conclusion for the topic.

\section{Security Requirements}
  % Gives an explanation of security again
  Information and network security are commonplace in the context of the Internet. Nevertheless, the \emph{IoT} poses several new conditions to the paradigm of networking, e.g. zillions of nodes participating, resource-constrained elements, huge loads of data being created. These considerations demand a closer look on the security approach that will be applied.

  Based the these underlying differences, Bassi, et al. \cite{Bassi2013} identify the following elements to be protected inside the \emph{IoT}: (1) The physical person; given that critical life-dependent services may be made unavailable, (2) the subject's privacy; being a subject both a person or a device), (3) the communications channel; to ensure, among others, data integrity, (4) leaf devices such as sensors, tags and actuators, (5) intermediary devices (e.g. gateways), (6) back-end services (e.g. data collection, sensor communication), (7) infrastructure services; these provide the required flexibility for end users to take advantage of the benefits of an \emph{IoT} platform, for example, service discovery, and finally (8) global systems; includes the whole \emph{IoT} platform.

  The \emph{Internet of Things Architecture} project (\emph{IoT-A}) \cite{Salinas2013}, states that aspects like data integrity and the trustworthiness of the services providing such information are key points for offering a reliable application. The platform has to offer confidentiality and privacy up to a certain extent to protect the interests of the users. Service availability becomes crucial when the involved functionalities are vital. Furthermore, operations under the context of the \emph{IoT} have to be authenticated in order to ensure many of the properties mentioned above. Additionally, non-repudiation is a desired property when protection against malicious users that deny having performed illegal actions inside the platform is desired. Interesting enough, in order to achieve non-repudiation, user information has to be revealed to a certain extent, leading to a trade-off between privacy and identification.

  % Introduces the 4-5 chosen requirements and remark their importance in the context of IoT 
  The chosen security requirements were selected in order to identify on which aspects \emph{AAA} plays a critical role. Nevertheless, by no means I imply that the only way to fulfill these requirements is through \emph{AAA} techniques. The selected requirements are as follows: (A) Trust was chosen to reflect the need of authentication in an \emph{IoT} platform by ensuring users that the collected data was originated from valid nodes and that the service users are who they say they are. (B) Confidentiality was chosen to denote the need of authorization in order to achieve different layers of protected information. (C) Non-repudiation was chosen to remark the importance of accountability, commonly neglected in non-critical scenarios. On the other hand, (D) privacy was chosen to contrast the trade-off between accountability and the impact on the end users.

  % Lists each requirement. Give a general definition and explain how they relate to the IoT
  % Trust as part of the authentication 
  \subsection{Trust \& Authentication}
  A definition of trust that fits the context of the \emph{IoT} is ``the extent to which one party is willing to participate in a given action with a given partner in a given situation, considering the risks and incentives involved'' \cite{Ruohomaa2006}. Trust becomes a fundamental security requirement when considering operations between multiple parties.

  Authentication defines a way to establish trust by verifying and validating the identity of the parties involved in an operation, thus, allowing them to bind. Any authentication process consists of two steps: first, the provision of evidence of authenticity and second, the validation of such a proof \cite{Sklavos2007}. This validation can be done by the other interested party or by a certification agent.

  Achieving trust in an \emph{IoT} platform requires overcoming a set of challenges related to the nature of the elements that conforms it. Information captured by leaf devices has to be accessed only by trustworthy services behind authenticated users; a breach in this regard may compromise private information or allow the execution of unauthorized action (e.g. tampering with health care devices). On the other hand, the platform has to be able to trust the information provided by the sensor networks ---hence the networks themselves---, resulting this in a challenge when dealing with the identities of the subjects \cite{Kanuparthi2013}. Resource-constrained elements should be able to perform security operations that enable them to authenticate without overwhelming their capacity. 

  % Confidentiality as part of the authorization
  \subsection{Confidentiality \& Authorization}
  Confidentiality focuses on preventing undesired elements to obtain or eavesdrop information or to gain access to restricted actions and privileges. The tampering of information flows must be prevented as well.

  Authorization and access control take place after the execution of an authentication protocol. This process determines whether the interested party has the sufficient privileges to perform certain actions. Access control can be achieved through different methods: Access Control List (ACL) which is a simple but not scalable approach, Role Based Access Control (RBAC) which maps roles to users in order to group sets of privileges, Attribute Based Access Control (ABAC) that aims to specify policies and access rules based on the attributes of the users  \cite{Anggorojati2014}. Another approach called Capability-Based Access Control (CBAC), where authorization tokens are generated and given to users, provides a dynamic authorization mechanism.

  Confidentiality can be enforced through access control policies that would protect the information from being accessed by unauthorized subjects. Moreover, confidentiality during the information flow processes is commonly achieved through cryptographic communication, however, elements participating in the \emph{IoT} suffer from different constraints that will prove this methods inefficient \cite{Baldini2012}.    

  % Non-repudiation/Identification as part of accountability
  % Privacy in contrast to accountability
  \subsection{Non-Repudiation, Privacy \& Accountability}
  Non-repudiation aims to protect the participants of a communication flow from having one of them denying its participation. This property makes sense when the operation on which the participants are involved is critical ---an electronic money transfer is a typical example. A certain degree of disclosure over the identities of the participants is needed in order to provide a service that conforms with non-repudiation. On the other hand, privacy enforces the confidentiality of the data. A participant has the solely right to control the access and use of its own information.

  The balance between these two contradictory properties has been defined as the degree of accountability in a system \cite{Bassi2013}. Accountability, in terms of a network, refers to the capability of collecting information on the usage of resources in a network. The recollection of the usage information depends on the accounting protocols ---data transmitted, time of usage, purpose, etc  \cite{Sklavos2007}. In the context of the \emph{IoT}, accounting can be used for different purposes. One of the most relevant is defining a criteria process for the use of resources \cite{Bauge2010}.

  Strict privacy enforcement is conflicting with aspects like authorization, authentication and non-repudiation. Through the use of fictional identities ---sometimes referred to as pseudonyms---, an \emph{IoT} platform can fulfill both non-repudiation and privacy requirements \cite{Baldini2012}. 
  
\section{Case Studies}
  % Introduce the section
  In this section I will introduce the three platforms to be assessed and will summarize their approaches towards the selected security requirements.

  % Name the chosen systems that will be discussed and explain why were they selected
  The chosen platforms were developed with the support of the \emph{Seventh Framework Programme for Research and Technological Development} of the \emph{European Commission} under the activity area of information and communication technologies \cite{FP7}\cite{FP7-ICT}\cite{ICT}. The documentation requirements for each platform were framed under the specifications indicated by the program.

  Although the three platforms engage on the idea of the \emph{IoT}, they have different goals. The selected platforms are: \emph{OpenIoT}, \emph{SENSEI}  and \emph{IoT@Work} \cite{OpenIoT} \cite{SENSEI} \cite{IoTWork}.

  \subsection{OpenIoT}

  The \emph{OpenIoT} platform \cite{OpenIoTWeb} consists of an open-source middleware under the LGPL 3.0 license that aims to provide services from sensor clouds encapsulating the exact sensors that are being used. This abstraction allows the dynamic integration of new services for collecting and providing information streams. The generated information is managed in a cloud-based approach that they named ``Sensing-as-a-Service'' \cite{Baldini2012}.  The framework masks the underlying semantical discrepancies ensuring interoperability in the technological stack and serving as a central coordinator in the exchange of data.

  Authentication and authorization in \emph{OpenIoT} is offered at the service level. This means that a user interested in collecting the data provided by a service is needed to be authenticated and authorized in order to access it. They achieve this by centralizing the process in Central Access Server (CAS) using a token-based approach with OAuth2.0. 

  New services are added to the platform using a security console bundled with the software. An administrator is able to define permissions and roles over user and services to restrict the access. Clients accessing the services can authenticate themselves via a web-interface or using web-services connecting to the CAS.

  To achieve the accounting process, the platform keeps track of several utility metrics on physical sensors, sensor networks and services. Aspects like energy consumption, bandwidth, data volume and trustworthiness are being measured on the physical sensors while in the case of services and sensor networks, aspects like the system lifetime, delays, bandwidth capacity, reliability and even confidentiality are evaluated. The gathering of this information is associated to a user of an \emph{OpenIoT} application. All this information is used to define the billing functionality \cite{Calbimonte}.

  \subsection{SENSEI}
  The SENSEI project \cite{SENSEIWeb} focused on integrating heterogeneous wireless sensors and actuators networks into a single, scalable and open framework that could withstand the demands of global businesses.
  
  In general terms, SENSEI architecture is divided in three layers: the application layer, the resource layer and the communication services layer \cite{Tsiatsis2010}. The application layer offers the necessary abstractions to allow end-business applications to access information streams and interaction functionality on real world entities via interfaces of the resource layer. The communication services layer abstracts all the underlying network issues allowing flexibility and adaptability for new technologies at higher levels. Finally, the resource layer manages the core functionality of the framework. 

  The resource layer is in charge of maintaining the discovery processes of the of real world entities; these entities are mapped to logical concepts named `resources' which encapsulate the physical properties of the devices. A resource can provide sensing and actuation behavior among others. A resource user communicates to a resource via a resource end point; a software that implements the required interfaces to interact with the resource. These resource end points are uniquely addressed inside a deployment and can be found through the discovery services. The abstraction resulting from this layer allows the coordination of mobility services in an flexible fashion \cite{Tsiatsis2010}. 
  
  SENSEI specifies an AAA framework tuned to the needs of wireless sensor networks, where resource-constrained nodes are common and short lived connections mandate. Its scope is to mutually authenticate parties within SENSEI, authorize the actions that they are allowed to do and keep track of these actions in order to be able to process payments for them. This framework relies on the concepts of tokens and policies to achieve its purposes in a secure manner \cite{Bauge2010}.

  Users require an account in the AAA framework in order to be able to access its services. After creating an account, a user can obtain security tokens that carry the necessary assertions to determine aspects such as account validity, role validation and user's identity. Policies allow making decisions based on the credentials of the users.

  In order to achieve interconnectivity between different framework providers, SENSEI establishes a centralized identity validation that allows the authentication token services from each counterpart to issue a valid token to be used in a framework different than its own. SENSEI combines this approach with an identity management ---such as pseudonymization--- thus, addressing the scalability and privacy concerns. 

  \subsection{IoT@Work}

  The \emph{IoT@Work} project aims to apply the advantages of \emph{IoT} technologies in the field of industrial automation, in particular manufacturing processes. Their goal is to use information resulting from even streams to reduce operational costs by minimizing the time spent in reconfiguration activities \cite{IoTWorkWeb}.

  The architecture is conformed by a combination of layers and planes. Layers make a functional separation of responsibilities while planes offer transversal services to all of the layers indistinctly. Three layers are defined: the \emph{device and network embedded services layer} in charge of device semantics and communication interfaces, the \emph{device resource creation \& management service} layer in charge of abstracting the device and network as well as providing the directory service, and the \emph{application level middleware services} layer in charge of composing, adapting and communicating the information to the applications. Three planes are defined as well, these are: the \emph{communication plane} in charge of managing the network aspects of several applications, the \emph{security plane} responsible of adding different security mechanism in the layers and the \emph{management plane} allowing service management on top of the existing infrastructure \cite{Rotondi2013}.

  Whenever a new device is incorporated to the system, an IPv6 address is assigned to it and afterwards, the directory service gets updated automatically with the means necessary to locate the given device. If such a device is capable of providing information, its information flow can be registered into a middleware component, called \emph{event notification system} or ENS, that interconnects event sources and consumers. 

  In the ENS, the resources produced by a device are arranged in hierarchical namespaces, with capability restrictions, that allow the organization of information flows for several purposes, for example, calculation of complex events resulting from multiple sources and association with physical entities such as assembly line robots \cite{Trsek2013}.

  \emph{IoT@Work} uses a capability-based access control approach to authenticate and authorize access to resources. During the registration of the resource in the ENS, a certain capability originated from the namespace is associated with the resource. These capabilities are communicated via authority tokens that uniquely identify an object and its set of access rights.

  The tokens are common for a certain namespace and are requested by the high level services. They can be represented using an XML format that will be signed with an X.509 certificate by the ENS. Authentication at the network level can be achieved via secure ids associated to each device in combination with a protocol like EAP.

\section{Evaluation}
  % TODO: Comment on the different approaches of the projects.
  This is just a phase.
  % TODO: Create a table comparing each of the selected security requirements
  \begin{table}[!t]    
    \renewcommand{\arraystretch}{1.3}    
    \caption{An Example of a Table}
    \label{evaluation_table}        
    \begin{tabu} to \columnwidth {X[0.6,l]X[0.5,c]X[,c]X[,c]X[0.3,c]}
      \hline      
                   & Trust      & Confidentiality & Non-Repudiation & Privacy \\
      \hline
      OpenIoT      & High       & Yes             & Yes             & No      \\
      SENSEI       & High       & Yes             & Yes             & Yes     \\
      IoT@Work     & High/Low   & Yes             & N/A             & N/A     \\ 
      \hline      
    \end{tabu}
  \end{table}
  % TODO: Consider speaking about Device Authentication as referred in Baldini
  

\section{Conclusion}
  % TODO: contrast the results and indicate why are they different or not, based on how their goals differ (probably)
  % TODO: Comment about how different requirements, specially on a big scale demand different AAA and security approaches.
  % TODO: Comment about the advantage of open source approach in OpenIoT for extending services compared to SENSEI

\bibliographystyle{IEEEtran}
\bibliography{surveyBib}

\end{document}/